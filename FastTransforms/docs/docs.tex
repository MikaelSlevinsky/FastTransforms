\documentclass{article}

\usepackage{amsmath,mathdots}

\def\ud{{\rm\,d}}
\def\ii{{\rm i}}
\def\fl{{\rm\,fl}}

\def\abs#1{\left|{#1}\right|}
\def\norm#1{\left\|{#1}\right\|}
\def\conj#1{\overline{#1}}


\begin{document}

\title{FastTransforms Documentation}

\author{Richard Mika\"el Slevinsky\thanks{Email: Richard.Slevinsky@umanitoba.ca}}

\maketitle

\section{Introduction}

{\tt FastTransforms} is a C library based on the solutions of the two-dimensional harmonic polynomial connection problems in~\cite{Slevinsky-ACHA-17,Slevinsky-1711-07866} that have an $\mathcal{O}(n^3)$ run-time, where $n$ is the polynomial degree, and are $2$-normwise backward stable.

\section{What {\tt FastTransforms} actually computes}

For every subsection below, there is a pre-computation called by the prefix \verb+plan_+, a forward execution called by the prefix \verb+execute_+, and a backward execution called by the prefix \verb+execute_a2b+ with the roles of \verb+a+ and \verb+b+ reversed.

\subsection{{\tt sph2fourier}}

Spherical harmonics are:
\begin{equation}
Y_\ell^m(\theta,\varphi) = \frac{e^{\ii m\varphi}}{\sqrt{2\pi}} \ii^{m+|m|}\sqrt{(\ell+\tfrac{1}{2})\frac{(\ell-m)!}{(\ell+m)!}} P_\ell^m(\cos\theta)
\end{equation}
where $P_\ell^m(\cos\theta)$ are the associated Legendre functions. A degree-$n$ expansion in spherical harmonics is given by:
\begin{equation}
f_n(\theta,\varphi) = \sum_{\ell=0}^{n}\sum_{m=-\ell}^{+\ell} f_\ell^m Y_\ell^m(\theta,\varphi).
\end{equation}
If spherical harmonic expansion coefficients are organized into the array:
\begin{equation}
F = \begin{pmatrix}
f_0^0 & f_1^{-1} & f_1^1 & f_2^{-2} & f_2^2 & \cdots & f_n^{-n} & f_n^n\\
f_1^0 & f_2^{-1} & f_2^1 & f_3^{-2} & f_3^2 & \cdots & 0 & 0\\
\vdots & \vdots & \vdots &  \vdots &  \vdots & \ddots & \vdots & \vdots\\
f_{n-2}^0 & f_{n-1}^{-1} & f_{n-1}^1 & f_n^{-2} & f_n^2 &  & \vdots & \vdots\\
f_{n-1}^0 & f_n^{-1} & f_n^1 & 0 & 0 & \cdots & 0 & 0\\
f_n^0 & 0 & 0 & 0 & 0 & \cdots & 0 & 0\\
\end{pmatrix},
\end{equation}
then {\tt sph2fourier} returns the bivariate Fourier coefficients:
\begin{equation}
G = \begin{pmatrix}
g_0^0 & g_0^{-1} & g_0^1 & \cdots & g_0^{-n} & g_0^n\\
g_1^0 & g_1^{-1} & g_1^1 & \cdots & g_1^{-n} & g_1^n\\
\vdots & \vdots & \vdots & \ddots & \vdots & \vdots\\
g_{n-1}^0 & g_{n-1}^{-1} & g_{n-1}^1& \cdots & g_{n-1}^{-n} & g_{n-1}^n\\
g_n^0 & 0 & 0 & \cdots & g_n^{-n} & g_n^n\\
\end{pmatrix}.
\end{equation}
That is:
\begin{equation}
g_n(\theta,\varphi) = \sum_{\ell=0}^n\sum_{m=-n}^{+n} g_\ell^m \frac{e^{\ii m\varphi}}{\sqrt{2\pi}} \left\{\begin{array}{lr} \cos(\ell\theta) & m~{\rm even},\\ \sin((\ell+1)\theta) & m~{\rm odd}.\end{array}\right.
\end{equation}
Since {\tt sph2fourier} only transforms columns of the arrays, the routine is indifferent to the choice of longitudinal basis; it may be complex exponentials or sines and cosines, with no particular normalization.

\subsection{{\tt tri2cheb}}

Triangular harmonics are:
\begin{equation}
\tilde{P}_{\ell,m}^{(\alpha,\beta,\gamma)}(x,y) = (2(1-x))^m \tilde{P}_{\ell-m}^{(2m+\beta+\gamma+1,\alpha)}(2x-1) \tilde{P}_m^{(\gamma,\beta)}\left(\frac{2y}{1-x}-1\right),
\end{equation}
where the tilde implies that the univariate Jacobi polynomials are orthonormal. A degree-$n$ expansion in triangular harmonics is given by:
\begin{equation}
f_n(x,y) = \sum_{\ell=0}^{n}\sum_{m = 0}^\ell f_\ell^m \tilde{P}_{\ell,m}^{(\alpha,\beta,\gamma)}(x,y).
\end{equation}
If triangular harmonic expansion coefficients are organized into the array:
\[
F = \begin{pmatrix}
f_0^0 & f_1^1 & f_2^2 & \cdots & f_n^n\\
f_1^0 & f_2^1 & f_3^2 & \cdots & 0\\
\vdots & \vdots &  \vdots & \iddots & 0\\
f_{n-2}^0 & f_{n-1}^1 & f_n^2 & \iddots & \vdots\\
f_{n-1}^0 & f_n^1 & 0 & \cdots & 0\\
f_n^0 & 0 & 0 & \cdots & 0\\
\end{pmatrix},
\]
then {\tt tri2cheb} returns the bivariate Chebyshev coefficients:
\[
G = \begin{pmatrix}
g_0^0 & g_0^1 & \cdots & g_0^n\\
g_1^0 & g_1^1 & \cdots & g_1^n\\
\vdots & \vdots & \ddots & \vdots\\
g_{n-1}^0 & g_{n-1}^1& \cdots & g_{n-1}^n\\
g_n^0 & 0 & \cdots & g_n^n\\
\end{pmatrix}.
\]
That is:
\[
g_n(x,y) = \sum_{\ell=0}^n\sum_{m=0}^n g_\ell^m T_\ell(2x-1) T_m\left(\frac{2y}{1-x}-1\right).
\]

%\bibliographystyle{unsrt}
%\bibliography{/Users/Mikael/Bibliography/Mik}
\begin{thebibliography}{1}

\bibitem{Slevinsky-ACHA-17}
R.~M. Slevinsky.
\newblock Fast and backward stable transforms between spherical harmonic
  expansions and bivariate {F}ourier series.
\newblock {\em Appl. Comput. Harmon. Anal.}, 2017.

\bibitem{Slevinsky-1711-07866}
R.~M. Slevinsky.
\newblock Conquering the pre-computation in two-dimensional harmonic polynomial
  transforms.
\newblock arXiv:1711.07866, 2017.

\end{thebibliography}

\end{document}