\documentclass{article}

\usepackage{amsmath,mathdots}

\def\ud{{\rm\,d}}
\def\ii{{\rm i}}
\def\fl{{\rm\,fl}}

\def\abs#1{\left|{#1}\right|}
\def\norm#1{\left\|{#1}\right\|}
\def\conj#1{\overline{#1}}


\begin{document}

\title{FastTransforms Documentation}

\author{Richard Mika\"el Slevinsky\thanks{Email: Richard.Slevinsky@umanitoba.ca}}

\date{}

\maketitle

\section{Introduction}

{\tt FastTransforms} is a C library based on the solutions of the two-dimensional harmonic polynomial connection problems in~\cite{Slevinsky-ACHA-17,Slevinsky-1711-07866} that have an $\mathcal{O}(n^3)$ run-time, where $n$ is the polynomial degree, and are $2$-normwise backward stable.

The transforms are separated into computational kernels that offer SSE, AVX, and AVX-512 vectorization on applicable Intel processors, and driver routines that are easily parallelized by OpenMP.

\section{What {\tt FastTransforms} actually computes}

For every subsection below, the title of the subsection, of the form \verb+a2b+, refers conceptually to the transform and the available functions are as follows:
\begin{itemize}
\item \verb+ft_plan_a2b+, is a pre-computation,
\item \verb+ft_execute_a2b+, is a forward execution,
\item \verb+ft_execute_b2a+, is a backward execution,
\item \verb+ft_execute_a_hi2lo+, is a conversion to a tensor-product basis,
\item \verb+ft_execute_a_lo2hi+, is a conversion from a tensor-product basis,
\item \verb+ft_kernel_a_hi2lo+, is an orthonormal conversion from high to low order,
\item \verb+ft_kernel_a_lo2hi+, is an orthonormal conversion from low to high order.
\end{itemize}
The \verb+ft_execute_*+ functions are drivers that perform transforms as defined below. They are composed of computational kernels, \verb+ft_kernel_*+ functions, that may be assembled differently.

\subsection{{\tt sph2fourier}}

Spherical harmonics are:
\begin{equation}
Y_\ell^m(\theta,\varphi) = \frac{e^{\ii m\varphi}}{\sqrt{2\pi}} (-1)^{\abs{m}}\sqrt{(\ell+\tfrac{1}{2})\frac{(\ell-\abs{m})!}{(\ell+\abs{m})!}} P_\ell^{\abs{m}}(\cos\theta),
\end{equation}
where $P_\ell^m(\cos\theta)$ are the associated Legendre functions. A degree-$n$ expansion in spherical harmonics is given by:
\begin{equation}
f_n(\theta,\varphi) = \sum_{\ell=0}^{n}\sum_{m=-\ell}^{+\ell} f_\ell^m Y_\ell^m(\theta,\varphi).
\end{equation}
If spherical harmonic expansion coefficients are organized into the array:
\begin{equation}
F = \begin{pmatrix}
f_0^0 & f_1^{-1} & f_1^1 & f_2^{-2} & f_2^2 & \cdots & f_n^{-n} & f_n^n\\
f_1^0 & f_2^{-1} & f_2^1 & f_3^{-2} & f_3^2 & \cdots & 0 & 0\\
\vdots & \vdots & \vdots &  \vdots &  \vdots & \iddots & \vdots & \vdots\\
f_{n-2}^0 & f_{n-1}^{-1} & f_{n-1}^1 & f_n^{-2} & f_n^2 &  & \vdots & \vdots\\
f_{n-1}^0 & f_n^{-1} & f_n^1 & 0 & 0 & \cdots & 0 & 0\\
f_n^0 & 0 & 0 & 0 & 0 & \cdots & 0 & 0\\
\end{pmatrix},
\end{equation}
then {\tt sph2fourier} returns the bivariate Fourier coefficients:
\begin{equation}
G = \begin{pmatrix}
g_0^0 & g_0^{-1} & g_0^1 & \cdots & g_0^{-n} & g_0^n\\
g_1^0 & g_1^{-1} & g_1^1 & \cdots & g_1^{-n} & g_1^n\\
\vdots & \vdots & \vdots & \ddots & \vdots & \vdots\\
g_{n-1}^0 & g_{n-1}^{-1} & g_{n-1}^1& \cdots & g_{n-1}^{-n} & g_{n-1}^n\\
g_n^0 & 0 & 0 & \cdots & g_n^{-n} & g_n^n\\
\end{pmatrix}.
\end{equation}
That is:
\begin{equation}
f_n(\theta,\varphi) = \sum_{\ell=0}^n\sum_{m=-n}^{+n} g_\ell^m \frac{e^{\ii m\varphi}}{\sqrt{2\pi}} \left\{\begin{array}{lr} \cos(\ell\theta) & m~{\rm even},\\ \sin((\ell+1)\theta) & m~{\rm odd}.\end{array}\right.
\end{equation}
Since {\tt sph2fourier} only transforms columns of the arrays, the routine is indifferent to the choice of longitudinal basis; it may be complex exponentials or sines and cosines, with no particular normalization.

\subsection{{\tt spinsph2fourier}}

Spin-weighted spherical harmonics are:
\begin{align}
Y_{\ell,m}^s(\theta,\varphi) & = \frac{e^{\ii m\varphi}}{\sqrt{2\pi}} \sqrt{(\ell+\tfrac{1}{2})\dfrac{(\ell+\ell_0)!(\ell-\ell_0)!}{(\ell+\ell_1)!(\ell-\ell_1)!}}\nonumber\\
& \times \sin^{\abs{m+s}}(\tfrac{\theta}{2})\cos^{\abs{m-s}}(\tfrac{\theta}{2}) P_{\ell-\ell_0}^{(\abs{m+s},\abs{m-s})}(\cos\theta).
\end{align}
where $P_n^{(\alpha,\beta)}(\cos\theta)$ are the Jacobi polynomials and $\ell_0 = \max\{\abs{m},\abs{s}\}$ and $\ell_1 = \min\{\abs{m},\abs{s}\}$. A degree-$n$ expansion in spin-weighted spherical harmonics is given by:
\begin{equation}
f_n^s(\theta,\varphi) = \sum_{\ell=\ell_0}^{n}\sum_{m=-\ell}^{+\ell} f_\ell^m Y_{\ell,m}^s(\theta,\varphi).
\end{equation}
If spin-weighted spherical harmonic expansion coefficients with $s=2$, for example, are organized into the array:
\begin{equation}
F = \begin{pmatrix}
f_2^0 & f_2^{-1} & f_2^1 & f_2^{-2} & f_2^2 & \cdots & f_n^{-n} & f_n^n\\
f_3^0 & f_3^{-1} & f_3^1 & f_3^{-2} & f_3^2 & \cdots & 0 & 0\\
\vdots & \vdots & \vdots &  \vdots &  \vdots & \iddots & \vdots & \vdots\\
f_{n}^0 & f_{n}^{-1} & f_{n}^1 & f_n^{-2} & f_n^2 &  & \vdots & \vdots\\
0 & 0 & 0 & 0 & 0 & \cdots & 0 & 0\\
0 & 0 & 0 & 0 & 0 & \cdots & 0 & 0\\
\end{pmatrix},
\end{equation}
then {\tt spinsph2fourier} returns the bivariate Fourier coefficients:
\begin{equation}
G = \begin{pmatrix}
g_0^0 & g_0^{-1} & g_0^1 & \cdots & g_0^{-n} & g_0^n\\
g_1^0 & g_1^{-1} & g_1^1 & \cdots & g_1^{-n} & g_1^n\\
\vdots & \vdots & \vdots & \ddots & \vdots & \vdots\\
g_{n-1}^0 & g_{n-1}^{-1} & g_{n-1}^1& \cdots & g_{n-1}^{-n} & g_{n-1}^n\\
g_n^0 & 0 & 0 & \cdots & g_n^{-n} & g_n^n\\
\end{pmatrix}.
\end{equation}
That is:
\begin{equation}
f_n(\theta,\varphi) = \sum_{\ell=0}^n\sum_{m=-n}^{+n} g_\ell^m \frac{e^{\ii m\varphi}}{\sqrt{2\pi}} \left\{\begin{array}{lr} \cos(\ell\theta) & m+s~{\rm even},\\ \sin((\ell+1)\theta) & m+s~{\rm odd}.\end{array}\right.
\end{equation}
Since {\tt spinsph2fourier} only transforms columns of the arrays, the routine is indifferent to the choice of longitudinal basis; it may be complex exponentials or sines and cosines, with no particular normalization.

\subsection{{\tt tri2cheb}}

Triangular harmonics are:
\begin{equation}
\tilde{P}_{\ell,m}^{(\alpha,\beta,\gamma)}(x,y) = (2-2x)^m \tilde{P}_{\ell-m}^{(2m+\beta+\gamma+1,\alpha)}(2x-1) \tilde{P}_m^{(\gamma,\beta)}\left(\frac{2y}{1-x}-1\right),
\end{equation}
where the tilde implies that the univariate Jacobi polynomials are orthonormal. A degree-$n$ expansion in triangular harmonics is given by:
\begin{equation}
f_n(x,y) = \sum_{\ell=0}^{n}\sum_{m = 0}^\ell f_\ell^m \tilde{P}_{\ell,m}^{(\alpha,\beta,\gamma)}(x,y).
\end{equation}
If triangular harmonic expansion coefficients are organized into the array:
\begin{equation}
F = \begin{pmatrix}
f_0^0 & f_1^1 & f_2^2 & \cdots & f_n^n\\
%f_1^0 & f_2^1 & f_3^2 & \cdots & 0\\
\vdots & \vdots &  \vdots & \iddots & 0\\
f_{n-2}^0 & f_{n-1}^1 & f_n^2 & \iddots & \vdots\\
f_{n-1}^0 & f_n^1 & 0 & \cdots & 0\\
f_n^0 & 0 & 0 & \cdots & 0\\
\end{pmatrix},
\end{equation}
then {\tt tri2cheb} returns the bivariate Chebyshev coefficients:
\begin{equation}
G = \begin{pmatrix}
g_0^0 & g_0^1 & \cdots & g_0^n\\
g_1^0 & g_1^1 & \cdots & g_1^n\\
\vdots & \vdots & \ddots & \vdots\\
%g_{n-1}^0 & g_{n-1}^1& \cdots & g_{n-1}^n\\
g_n^0 & g_n^1 & \cdots & g_n^n\\
\end{pmatrix}.
\end{equation}
That is:
\begin{equation}
f_n(x,y) = \sum_{\ell=0}^n\sum_{m=0}^n g_\ell^m T_\ell(2x-1) T_m\left(\frac{2y}{1-x}-1\right).
\end{equation}

\subsection{{\tt disk2cxf}}

Disk harmonics are Zernike polynomials:
\begin{equation}
Z_\ell^m(r,\theta) = \sqrt{2\ell+2} r^{\abs{m}}P_{\frac{\ell-\abs{m}}{2}}^{(0,\abs{m})}(2r^2-1)\frac{e^{\ii m\theta}}{\sqrt{2\pi}}.
\end{equation}
A degree-$2n$ expansion in disk harmonics is given by:
\begin{equation}
f_{2n}(r,\theta) = \sum_{\ell=0}^{2n}\sum_{m=-\ell,2}^{+\ell} f_\ell^m Z_\ell^m(r,\theta),
\end{equation}
where the $,2$ in the inner summation index implies that the inner summation runs from $m=-\ell$ in steps of $2$ up to $+\ell$. If disk harmonic expansion coefficients are organized into the array:
\begin{equation}
F = \begin{pmatrix}
f_0^0 & f_1^{-1} & f_1^1 & f_2^{-2} & f_2^2 & \cdots & f_{2n}^{-2n} & f_{2n}^{2n}\\
f_2^0 & f_3^{-1} & f_3^1 & f_4^{-2} & f_4^2 & \cdots & 0 & 0\\
\vdots & \vdots & \vdots &  \vdots &  \vdots & \iddots & \vdots & \vdots\\
f_{2n-4}^0 & f_{2n-3}^{-1} & f_{2n-3}^1 & f_{2n-2}^{-2} & f_{2n-2}^2 &  & \vdots & \vdots\\
f_{2n-2}^0 & f_{2n-1}^{-1} & f_{2n-1}^1 & f_{2n}^{-2} & f_{2n}^2 & \cdots & 0 & 0\\
f_{2n}^0 & 0 & 0 & 0 & 0 & \cdots & 0 & 0\\
\end{pmatrix},
\end{equation}
then {\tt disk2cxf} returns the even Chebyshev--Fourier coefficients:
\begin{equation}
G = \begin{pmatrix}
g_0^0 & g_0^{-1} & g_0^1 & g_0^{-2} & g_0^2 & \cdots & g_0^{-2n} & g_0^{2n}\\
g_2^0 & g_2^{-1} & g_2^1 & g_2^{-2} & g_2^2 & \cdots & g_2^{-2n} & g_2^{2n}\\
\vdots & \vdots & \vdots & \vdots & \vdots & \ddots & \vdots & \vdots\\
g_{2n-2}^0 & g_{2n-2}^{-1} & g_{2n-2}^1 & g_{2n-2}^{-2} & g_{2n-2}^2 & \cdots & g_{2n-2}^{-2n} & g_{2n-2}^{2n}\\
g_{2n}^0 & 0 & 0 & g_{2n}^{-2} & g_{2n}^2 & \cdots & g_{2n}^{-2n} & g_{2n}^{2n}\\
\end{pmatrix}.
\end{equation}
That is:
\begin{equation}
f_{2n}(r,\theta) = \sum_{\ell=0}^{n}\sum_{m=-2n}^{+2n} g_{2\ell}^m \frac{e^{\ii m\theta}}{\sqrt{2\pi}} \left\{\begin{array}{lr} T_{2\ell}(r) & m~{\rm even},\\ T_{2\ell+1}(r) & m~{\rm odd}.\end{array}\right.
\end{equation}
Since {\tt disk2cxf} only transforms columns of the arrays, the routine is indifferent to the choice of azimuthal basis; it may be complex exponentials or sines and cosines, with no particular normalization.

\subsection{{\tt tet2cheb}}

Tetrahedral harmonics are:
\begin{align}
\tilde{P}_{k,\ell,m}^{(\alpha,\beta,\gamma,\delta)}(x,y,z) & = (2-2x)^{\ell+m} \tilde{P}_{k-\ell-m}^{(2\ell+2m+\beta+\gamma+\delta+2,\alpha)}(2x-1)\nonumber\\
& \quad \times \left(2-\frac{2y}{1-x}\right)^m \tilde{P}_\ell^{(2m+\gamma+\delta+1,\beta)}\left(\frac{2y}{1-x}-1\right)\\
& \quad \times \tilde{P}_m^{(\delta,\gamma)}\left(\frac{2z}{1-x-y}-1\right),\nonumber
\end{align}
where the tilde implies that the univariate Jacobi polynomials are orthonormal. A degree-$n$ expansion in tetrahedral harmonics is given by:
\begin{align}
f_n(x,y,z) & = \sum_{k=0}^n\sum_{\ell=0}^k\sum_{m=0}^{k-\ell} f_{k,\ell}^m \tilde{P}_{k,\ell,m}^{(\alpha,\beta,\gamma,\delta)}(x,y,z),\\
& = \sum_{m=0}^n\sum_{\ell=0}^{n-m}\sum_{k=\ell+m}^n f_{k,\ell}^m \tilde{P}_{k,\ell,m}^{(\alpha,\beta,\gamma,\delta)}(x,y,z).
\end{align}
If tetrahedral harmonic expansion coefficients are organized into the rank-$3$ array whose $m^{\rm th}$ slice is:
\begin{equation}
F[0:n-m,0:n-m,m] = \begin{pmatrix}
f_{m,0}^m & f_{1+m,1}^m & f_{2+m,2}^m & \cdots & f_{n,n-m}^m\\
\vdots & \vdots &  \vdots & \iddots & 0\\
f_{n-2,0}^m & f_{n-1,1}^m & f_{n,2}^m & \iddots & \vdots\\
f_{n-1,0}^m & f_{n,1}^m & 0 & \cdots & 0\\
f_{n,0}^m & 0 & 0 & \cdots & 0\\
\end{pmatrix},
\end{equation}
then {\tt tet2cheb} returns the trivariate Chebyshev coefficients stored in the rank-$3$ array $G$. That is:
\begin{equation}
f_n(x,y,z) = \sum_{k=0}^n\sum_{\ell=0}^n\sum_{m=0}^n g_{k,\ell}^m T_k(2x-1) T_\ell\left(\frac{2y}{1-x}-1\right) T_m\left(\frac{2z}{1-x-y}-1\right).
\end{equation}

\begin{thebibliography}{1}

\bibitem{Slevinsky-ACHA-17}
R.~M. Slevinsky.
\newblock Fast and backward stable transforms between spherical harmonic
  expansions and bivariate {F}ourier series.
\newblock {\em Appl. Comput. Harmon. Anal.}, 2017.

\bibitem{Slevinsky-1711-07866}
R.~M. Slevinsky.
\newblock Conquering the pre-computation in two-dimensional harmonic polynomial
  transforms.
\newblock arXiv:1711.07866, 2017.

\end{thebibliography}

\end{document}